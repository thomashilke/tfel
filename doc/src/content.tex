\section{Maillages et structures de données}

Le template \texttt{finite\_element\_space} est responsable de la
construction du mapping entre les indices des degrés de liberté
locaux à chaque élément, et les indices des degres de liberté
globaux.

Soit $M = (V, E)$ un mesh de triangles conformes. On note:
\begin{itemize}
\item $V = \{v_1, v_2, \dots, v_{N_v}\}$ l'ensemble des vertices du
  mesh $M$.

\item $\tau = \{K_1, K_2, \dots, K_{N_K}\}$, où $K_i = \{K_{i,1},
  K_{i,2}, K_{i,3}\}$, $K_{i,j}\in V$ l'ensemble des éléments du
  mesh $M$. On suppose que $\forall i$ les éléments de $e_i$ sont triés par
  ordre croissants, c'est-à-dire que $K_{i,1} < K_{i,2} < K_{i,3}$.

\item $A = \left\{ \{v, w\}\ |\ v,w\in V \text{ et } \exists e\in E
  \text{ tel que } v,w \in e\right\}$. On note $a_i = \{a_{i,1},
  a_{i,2}\}$, $i = 1, \dots, N_{a}$ les éléments de l'ensemble $A$,
  avec $N_a = \#A$. De même, on suppose que $\forall i$ les éléments de
  $a_i$ sont triés par ordre croissants, c'est-à-dire
  que $a_{i,1} < a_{i,2}$.
\end{itemize}

D'un point de vue de l'implémentation informatique, l'ensemble $V$
est représenté simplement par l'entier $N_v$. Dans un cadre plus
général, $V$ peut être représenté par une liste d'entiers
unique ${n_1, \dots, n_{N_v}}$, par exemple si la numérotation des
vertices ne commence pas à 1, ou que des intervalles manquent.

L'ensemble $E$ est représenté par un tableau à double entrées
de dimensions $N_e \times 3$:
\begin{addmargin}[0.5in]{0em}
  \texttt{unsigned int elements[n\_e][3];}
\end{addmargin}
où \texttt{elements[i][j]}$ = e_{i,j}$.

L'ensemble $A$ est représenté par un tableau à double entrées
de dimensions $N_a \times 2$:
\begin{addmargin}[0.5in]{0em}
  \texttt{unsigned int edges[n\_a][2];}
\end{addmargin}
où \texttt{edges[i][j]}$ = a_{i,j}$.


\section{Éléments finis et degrés de liberté}
Notons $\hat K$ le triangle de sommets $v_1 = (0,0)$, $v_2 = (1,0)$ et
$v_3 = (0, 1)$. On note $\hat N_v = 3$ le nombre de sommets du
triangle, $\hat N_K = 1$ le nombre de triangle et $\hat N_a = 3$ le
nombre d'arrête du triangle. Cette notation, qui parait superflue,
devient nécessaire lorsque l'on géneralise le formalisme à des
mesh de simplexes de dimensions $D \neq 2$.

Dans ce cas, l'ensemble $A$ est donné par
\begin{equation}
  \tau = \left\{ \{v_1, v_2\}, \{v_1, v_3\}, \{v_2, v_3\} \right\}.
\end{equation}

Un élément fini définit un certain nombre de degrés de
libertés sur chaque élément. En particulier, on note $\hat M_v$ le
nombre de degré de liberté associés à chaque vertex de $K$,
$\hat M_a$ le nombre de dof associés à chaque arrête de $K$, et
$\hat M_K$ le nombre de dof associés à l'élément $K$. Donc le
nombre de dof $\hat M_\text{tot}$ associés à $K$ au total est donné par
\begin{equation}
  \hat M_\text{tot} = \hat M_v \hat N_v + \hat M_a \hat N_a + \hat M_K \hat N_K.
\end{equation}
Dans un mesh les degrés de libertés associés aux vertices et aux
arrêtes d'un élément sont identifiés aux degrés de liberté
associés a un autre élément qui partage le même vertex ou la
même arrête. Ces dofs sont donc associés aux vertices et aux
arrêtes du maillages, plutôt que propres à un élément ou
l'autre.

Cependant, les deux points de vues sont utiles, puisque l'on assemble
la matrice élément fini élément par élément, et non fonction de
base après fonction de base.

La contribution d'un élément $K$ au système linéaire est d'abord
calculée en utilisant une numérotation des degrés de liberté
locale à $K$, puis la contribution est assemblée dans la matrice
globale. Pour cette étape, il est nécessaire d'identifier les
degrés de libertés locaux aux degrés de libertés globaux. C'est
le rôle de l'espace élément fini, décrit plus en détail dans
la section qui suit.

\section{Espace éléments finis}
On suppose donné un mesh $M$ un élément fini, c'est-à-dire que
$\hat M_v$, $\hat M_a$ et $\hat M_K$ sont donnés. Commen\c cons par
enumérer l'ensemble des degrés de libertés $M_\text{tot}$
associés au maillage:
\begin{align}
  & M_v = \hat M_v N_v\\
  & M_a = \hat M_a N_a\\
  & M_K = \hat M_K N_K\\
  & M_\text{tot} = \hat M_v N_v + \hat M_a N_a + \hat M_K N_K.
\end{align}

On note les degrés de liberté $\{\varphi_i\}_{i=1}^{M_\text{tot}}$,
et on associe les trois intervalle d'indices avec les degrés
associés respectivement aux vertices, aux arrêtes et aux
éléments:

\begin{itemize}
\item Indices des dofs associés aux vertices: $\{1, \dots, \hat M_v
  N_v\}$,
\item Indices des dofs associés aux arrêtes: $\{\hat M_v N_v + 1,
  \hat M_v N_v + \hat M_a N_a\}$,
\item Indices des dofs associés aux éléments: $\{\hat M_v N_v + \hat
  M_a N_a + 1, \dots, \hat M_v N_v + \hat M_a N_a + \hat M_K N_K\}$,
\end{itemize}

En particulier, le $\hat i$-ème dof associé au vertex $j$ a l'indice global
\begin{equation}
  i = j \hat M_v + \hat i,
\end{equation}
le $\hat i$-ème dof associé à l'arrête $j$ a l'indice global
\begin{equation}
  i = \hat M_v N_v + j \hat M_a + \hat i,
\end{equation}
and the $\hat i$-ème dof associé au triangle j a l'indice global
\begin{equation}
  i = \hat M_v N_v + \hat M_a N_a + j \hat M_K + \hat i.
\end{equation}

\section{Cellule de référence}
On définit ici la cellule (triangle) de référence, et en
particulier les enumeration des differents sous-domaines (vertices,
arrêtes et triangle). Soit les vertices $v_0 = (0,0)$, $v_1 =
(1,0)$ and $v_2 = (0,1)$, et le triangle est definit par $K = \{v_0,
v_1, v_2\}$. Donc on définit et enumère les sous domaines de la
manière suivante:
\begin{align*}
  \text{vertices: } & v_0, v_1, v_2,\\
  \text{arrêtes: } & e_0 = \{v_0, v_1\},\ e_1 = \{v_0, v_2\},\ e_2 =
  \{v_1, v_2\}, \\
  \text{triangle: } & K = \{v_0, v_1, v_2\}.
\end{align*}

Il y a quand meme un truc. Il faut que les dof definis sur les
arrêtes soient enumérés dans le même ordre sur chaque arrête de
la cellule de référence. La question ne se pose pas pour les dofs
définis sur les éléments, parce qu'ils ne sont par partagés par
différents éléments. Pour les dofs définis sur les vertices, la
question ne se pose pas, parce qu'il y a qu'une seule possibilité
d'enumérer un unique objet (le vertex).

\section{Assemblage}
On considère le problème suivant. Soit $V_h$ l'espace élément finit lagrange $\mathbb P_1\bigcap H^1_0(\Omega)$ sur une triangulation $\tau_h$ de $\Omega$. On cherche la fonction $u_h\in V_h$ telle que:
\begin{align}
  \int_\Omega \nabla u_h\cdot \nabla v = \int_\Omega f v,\quad \forall v \in V_h.
\end{align}
Bien entendu, on prend $v = \phi_j$ les fonctions de base de $V_h$, et on décompose les intégrale sur chaque élément de la triangulation:
\begin{equation}
  \sum_{K\in\tau_h}\int_K \nabla u_h\cdot \nabla v = \sum_{K \in \tau_h}\int_\Omega f v, \quad \forall v \in V_h.
\end{equation}
L'inconnue $u_h$ s'écrit comme une combinaison linéaire des fonctions de base $\phi_i$:
\begin{equation}
  u_h(x) = \sum_{i = 1}^{M_\text{tot}} u_i \phi_i(x).
\end{equation}
Le système devient alors:
\begin{equation}
  \sum_{i = 1}^{M_\text{tot}}u_i\sum_{K\in\tau_h}\int_K \nabla \phi_i\cdot \nabla \phi_j = \sum_{K \in \tau_h}\int_\Omega f \phi_j, \quad \forall j \in \{1, \dots,M_\text{tot}\}.
\end{equation}

Puisque le support des $\phi_j$ sont limités aux éléments qui partagent le degré de liberté $j$, la plupart des intégrales sont nulles. Soit $a$ la forme bilinéaire définie par
\begin{equation}
a(u, v) = \sum_{K\in\tau_h}\int_K \nabla u\cdot \nabla v.
\end{equation}
On veut calculer l'ensemble des nombres $a(\phi_i, \phi_j)$:
\begin{equation}
a(\phi_i, \phi_j) = \sum_{K\in\tau_h}\int_K \nabla \phi_i\cdot \nabla \phi_j = \sum_{K\in\tau_h}\int_{\hat K} \det(J(\hat x)) [J^{-\mathrm t}(\hat x)]_{kn} [J^{-\mathrm t}(\hat x)]_{km} \hat\partial_n \hat\phi_i(\hat x)\hat\partial_m \hat\phi_j(\hat x).
\end{equation}
En pratique, les intégrales sont remplacées par des quadratures numériques. Donc la forme bilinéaire $a$ devient:
\begin{equation}
a(\phi_i, \phi_j) = \sum_{K\in\tau_h} Q \sum_{q = 1}^{n_q}\omega_q \det(J(\hat x_q)) [J^{-\mathrm t}(\hat x_q)]_{kn} [J^{-\mathrm t}(\hat x_q)]_{km} \hat\partial_n \hat\phi_i(\hat x_q)\hat\partial_m \hat\phi_j(\hat x_q),
\end{equation}
avec $Q$ un coefficient de normalisation de la formule de quadrature, $\omega_q$ les poids de la formule de quadrature, et $\hat x_q$ les points de quadrature dans l'élément de référence.

\section{Formules de quadrature}
\begin{figure}
  \begin{center}
    \input{../figures/quadrature-convergence.tex}
    \caption{Convergence de l'erreur de différentes formules de quadrature. La quantité $I$ est définie par $\int_0^2 f(x)\,\mathrm dx$, avec $f(x) = \sin(x)$.}
    \label{fig:quadrature-convergence-1d}
  \end{center}
\end{figure}



\section{Éléments finis}
L'élément fini Lagrange continu et polynomial d'ordre 2 par morceau
sur des triangles est définit dans le tableau qui suit.

\begin{tabularx}{\textwidth}{@{}llllll@{}}
  \toprule
  DOF & Sousdomaine & Position & $\phi_i$ & $\partial_1\phi_i$
& $\partial_2\phi_i$ \\
  \midrule
  1 & vertex $1$     & $(0, 0)$     & $2(x_0 + x_1 - 1)(x_0 + x_1 - 0.5)$  & $4(x_0 + x_1) - 3$  & $4(x_0 + x_1) - 3$ \\
  2 & vertex $2$     & $(1, 0)$     & $2x_0(x_0 - 0.5) $                  & $4x_0 - 1$          & $0$ \\
  3 & vertex $3$     & $(0, 1)$     & $2x_1(x_1 - 0.5) $                  & $0$                 & $4x_1 - 1$ \\
  4 & arrête $(1,2)$ & $(0.5, 1)$   & $-4x_0(x_0 + x_1 - 1) $             & $-4(2x_0+x_1 - 1)$  & $-4x_0$ \\
  5 & arrête $(1,3)$ & $(1, 0.5)$   & $-4x_1(x_0 + x_1 - 1) $             & $-4x_1$             & $-4(x_0 + 2x_1 - 1)$ \\
  6 & arrête $(2,3)$ & $(0.5, 0.5)$ & $4x_0x_1 $                          & $4x_1$              & $4x_0$ \\
  \bottomrule
\end{tabularx}
